\cappar Olá! Chamo-me Mika Matías. Atualmente, estou a terminar os meus estudos na Universidade de Lusiadas Angola em Gestão de Recursos Humanos, e também trabalho para a firma que promove esta revista. A minha experiência nesta empresa é muito produtiva, uma vez que estou a obter um ótimo conhecimento da especialidade da empresa. A importância do trabalho que a nossa instituição tem realizado em águas e energia tem melhorado a qualidade de vida da população. Cada dia aprendo mais junto dos colegas do trabalho.

\vspace{0.5cm}
\begin{wrapfigure}{r}{0.60\textwidth} 
  \vspace{-25pt}
  \begin{figurebox}
   \vspace{20pt}
    \centering
    \includegraphics[height=0.5\textheight]{Mika.jpg}\\
    Mika Matias\\ 
     {\sl\small Recursos Humanos, Saema}
    \vspace{1pt}
    %\vspace{0.1\textheight}
  \end{figurebox}
 \vspace{-20pt}
\end{wrapfigure}

Queria falar da situação de Angola do meu ponto de vista. Angola é um país que obteve a sua independência em 1975 depois de mais de 500 anos de colonização portuguesa. A 11 de novembro de 1975, foi proclamada a independência de Angola pelo Dr. António Agostinho Neto. 
Todos os setores da economia angolana e das infraestruturas têm sido objeto dos esforços de revitalização, as comunicações da atualidade, como estradas, caminhos de ferro, portos e aeroportos, encontram-se em reabilitação e construção em toda Angola, de maneira a contribuir para a circulação de pessoas e mercadorias. Sem dúvida, Angola foi benzida pela natureza. É difícil saber qual é a sua maior riqueza. O seu povo, com uma grande 
diversidade cultural, a sua beleza e riqueza natural. As pessoas amáveis e hospitaleiras. Uma incrível variedade de paisagens e uma cultura dinâmica e cativante, fazem com que Angola seja um país que vale a pena explorar: os nossos rios, os nossos lagos, as nossas montanhas, os nossos bosques.

 Um dos principais capitais de Angola é, sem dúvida, a sua cultura em todas as suas manifestações. A música angolana quer 
tradicional (Semba , tarrachinha), quer moderna (Kizomba, Kuduro, Zouk) tem sido capaz de percorrer o seu caminho, já com uma importante
projeção internacional. Angola tem, sem dúvida nenhuma, muito a oferecer. Em termos de recursos naturais destacam-se, em primeiro lugar, o petróleo e o diamante, que atualmente suportam a economia angolana, mas há uma ampla gama de recursos, na sua maioria sem explorar, como o ferro, cobre, gás natural, madeiras preciosas, entre outros. Hoje, o governo de Angola centra-se na formação de quadros nacionais, principalmente na construção das universidades académicas existentes nas regiões dos países.

Gostaria de transmitir uma mensagem a todos os angolanos de Cabinda a Cunene, e o Mar de Leste, que unem as suas forças, ao trabalharem juntos pelo desenvolvimento do país, para a conquista dos níveis cimeiros no pódio da economia mundial. Este país é o seu, o meu, o nosso, é de todos os angolanos.

\newpage

%%% Local Variables: 
%%% mode: latex
%%% TeX-master: "solucionenaccion"
%%% End: 



